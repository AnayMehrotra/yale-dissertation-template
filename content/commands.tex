%%%%%%%%%%%%%%%%%%%%%%%
% Mark up
%%%%%%%%%%%%%%%%%%%%%%%
\newcommand{\fillin}[1]{\textcolor{red}{\textbf{#1}}}

%%%%%%%%%%%%%%%%%%%%%%%%
% Dissertation metadata
%%%%%%%%%%%%%%%%%%%%%%%%
%
% Define macros for the dissertation title, author name and year.  By defining
% these values once in this file they can be reused throughout the
% dissertation.  To change the title, author or year simply edit the
% definitions below – you should not have to search through the individual
% chapter files.  See content/title-page.tex, content/abstract.tex and
% content/copyright.tex for examples of how these macros are used.

% The main title of the dissertation.  If you wish to introduce manual
% line breaks in the title you can include ``\\'' within the definition.
\newcommand{\DissertationTitle}{Algorithmic Advances for the \\\\ Design and Analysis of Being a Good Boy}

% The full name of the author.  This appears on the title page, abstract and
% copyright page.  Avoid additional formatting here – formatting should be
% applied where the macro is used.
\newcommand{\AuthorName}{Handsome Dan}

% The year in which the dissertation is submitted.  This value is used on
% the abstract and copyright pages, and on the title page.
\newcommand{\DissertationYear}{20XX}